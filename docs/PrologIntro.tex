\secrel{\prolog\ intro}\secdown

The best tutorial on using \prolog\ you can found is
\href{https://www.cpp.edu/~jrfisher/www/prolog_tutorial/contents.html}{J.R.Fisher's
tutorial}.

\bigskip
\prolog\ can be very complex to understand especially for people already have
some experience in other imperative or functional language. The key let you
understand prolog\ --- think about clauses like hypergraph elements:
\begin{itemize}
  \item 
every term defines named relation between it's arguments.
\end{itemize}

\clearpage
For first steps you can run \flora\ directly from command line, and load sample
files using
\lst{direct \flora\ run from command line \linux}{src/linflora.rc}
\lst{direct \flora\ run from command line \win}{src/winflora.rc}

\secrel{Load a program from a local file}

\lst{(re)load this from local \file{.flr}\ file
using command \file{[hello].}}{src/hello.flr}

\begin{verbatim}
flora2 ?- [hello].
Hello World

Times (in seconds): elapsed = 0.0000; pure CPU = 0.0000

Yes
\end{verbatim}

On \win\ you can use \keys{\arrowkeyup}\ to reload last \verb$[hello].$

\secrel{.init file}

You can create dotfile\note{file with name starts from .dot traditionally used
in UNIX systems as config files (resides in user \file{\~{}/home} directory)}

\secrel{Comments}

\flora\ uses \cpp-like comments:
\begin{verbatim}
// line comment
/* block comment */
\end{verbatim}

\clearpage
\secrel{\prog{factorial}\ sample}

\lstx{\prog{factorial}\ in \prolog}{src/factorial.pl}{Prolog}
\lst{hypergraph representation}{src/factorial.hyp}

\secup
