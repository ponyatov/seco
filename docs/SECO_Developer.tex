% Universal LaTeX headers for e-book publications
\documentclass[oneside,10pt]{book}
%% mobile phone optimized papaer size
\usepackage[paperwidth=118.8mm,paperheight=68.2mm,margin=2mm]{geometry}
%% hyperlinks pdf style
\usepackage[unicode,colorlinks=true]{hyperref}
%% fix first title page
\usepackage{atbegshi}\AtBeginDocument{\AtBeginShipoutNext{\AtBeginShipoutDiscard}}
%% font setup for screen reading
\renewcommand{\familydefault}{\sfdefault}\normalfont
%% fix heading styles for tiny paper
\usepackage{titlesec}
\titleformat{\chapter}{\Large\bfseries}{\thechapter.}{1em}{}
\titleformat{\section}{\large\bfseries}{\thesection.}{1em}{}
%% fix contents style
\usepackage{tocloft}
\renewcommand{\cfttoctitlefont}{\Large}
\renewcommand{\cftbeforetoctitleskip}{0mm}
\renewcommand{\cftaftertoctitleskip}{1em}

\renewcommand{\cftchapfont}{\bfseries}
\renewcommand{\cftbeforechapskip}{0.2em}

\renewcommand{\cftsecfont}{}
\renewcommand{\cftsubsecfont}{\it}
\renewcommand{\cftparskip}{0mm}
%% fix heading styles for tiny paper
\usepackage{titlesec}
\titleformat{\chapter}{\Large\bfseries}{\thechapter.}{1em}{}
\titlespacing{\chapter}{0em}{0em}{0.5em}
\titleformat{\section}{\large\bfseries}{\thesection.}{1em}{}
\titlespacing{\section}{0em}{1em}{0.5em}
\titleformat{\subsection}{\large\bfseries}{\thesubsection.}{1em}{}
\titlespacing{\subsection}{0em}{1em}{0.5em}

% images
\usepackage[pdftex]{graphicx}
\newcommand{\fig}[2]{\noindent\includegraphics[#2]{#1}}

% relative sectioning
\usepackage{ifthen}
\newcounter{secdepth}\setcounter{secdepth}{0}
\newcommand{\secup}{\addtocounter{secdepth}{1}}
\newcommand{\secdown}{\addtocounter{secdepth}{-1}}
\newcommand{\secrel}[1]{
\ifthenelse{\equal{\value{secdepth}}{0}}{\part{#1}}{}
\ifthenelse{\equal{\value{secdepth}}{-1}}{\chapter{#1}}{}
\ifthenelse{\equal{\value{secdepth}}{-2}}{\section{#1}}{}
\ifthenelse{\equal{\value{secdepth}}{-3}}{\subsection{#1}}{}
\ifthenelse{\equal{\value{secdepth}}{-4}}{\subsubsection{#1}}{}
}
\newcommand{\secp}[1]{\paragraph{#1}}
\newcommand{\secly}[1]{
\section*{#1}
\addcontentsline{toc}{section}{#1}
}

% [nosep] option in lists/enums
\usepackage{enumitem}
% frame box
\usepackage{framed}

% misc
\newcommand{\email}[1]{$<$\href{mailto:#1}{#1}$>$}
\renewcommand{\emph}[1]{\textcolor{red}{#1}}
\newcommand{\note}[1]{\,\footnote{\ #1}}

% listing & computer
\usepackage[os=win]{menukeys}
\usepackage{amssymb} % windows key
\newcommand{\winstart}{$\boxplus$}
\newcommand{\winr}{\keys{\winstart+R}}
\newcommand{\lms}{$\lhd$}
% \newcommand{\dblms}{$\lhd\lhd$}
\newcommand{\rms}{$\rhd$}
% \newcommand{\checkbox}{$\boxtimes$}
% \newcommand{\uncheckbox}{$\square$}

%% languages
\newcommand{\seco}{$Seco$}
\newcommand{\hgdb}{$HGDB$}
\newcommand{\java}{$Java$}

%% OSes
\newcommand{\linux}{$Linux$}
\newcommand{\win}{$Windows$}

%% objects
\newcommand{\file}[1]{\textbf{#1}}

%%%%%%%%%%%%%%%%%%%%%%%%%%%%%%%%%%%%%%%%%%%%%%%%%%%%%%%%%%%%%%%%%%%%%%%%

\author{\copyright\ Dmitry Ponyatov \email{dponyatov@gmail.com}}
\title{\ \\\ \\Become a Seco Developer\\newbie guide}
\begin{document}
\maketitle
\tableofcontents\secdown
\clearpage

\secly{Intro to Seco IDE}

\seco\ is a collaborative scripting development environment for the \java\
platform. You can write code in many JVM scripting languages. The code editor in
Seco is based on the Mathematica notebook UI, but the full GUI is richer and
much more ambitious. In a notebook, you can mix rich text with code and output,
including interactive components created by your code. This makes Seco into a
live environment because you can evaluate expression and immediately see the
changes to your program.

\begin{description}
\item[GitHub]\ \\
\url{https://github.com/bolerio/seco}
\item[binary release builds]\ \\
\url{https://github.com/bolerio/seco/releases/latest}
\end{description}

You can use \seco\ from prebuild binary archive, but this manual will guide you
to become a developer able to expand \seco\ in ways you want.

\bigskip
This manual appeared when I found
\href{http://www.hypergraphdb.org/}{hypergraphdb} usable in my AI and knowledge
database research. \seco\ was noted as visual tool for experimenting with \hgdb,
and I was impressed by it's compact design. But \seco

\begin{description}[nosep]
\item[\href{https://github.com/bolerio/seco/issues/46}{issue 46}]
lacks of extension for \href{http://xsb.sourceforge.net/}{XSB Prolog} and
\href{http://coherentknowledge.com/comparison-of-ergo-suite-to-flora-2/}{Ergo/Flora}
system and
\item[\href{https://github.com/bolerio/seco/issues/45}{issue 45}]
has some problems and low usability with visual interface.
\end{description}

\bigskip
So I started some development without any notable experience \java\ development,
and add this manual going this way. You can track me on github, but note this is
fork project, and my modifications may be merged into core project:
\url{https://github.com/ponyatov/seco}

\secrel{Fresh taste}\secdown

For first time let's play with prebuild binary release: you will get feel of
Seco without long SDK installation.

\secrel{Workbench install} \secdown

\secrel{Java}

Prebuild Seco binary distro requires only \java\ RE, but we will install SDK:

\begin{enumerate}[nosep]
\item \url{http://www.oracle.com/technetwork/java/javase/downloads/}
\item \menu{JDK>Download}
\item \menu{Java SE Development Kit 8u131>Accept License Agreement}
\item \menu{jdk-8u131-windows-i586.exe} 32 bit (win32)\\
\menu{jdk-8u131-linux-x64.tar.gz} 64 bit
\end{enumerate}

\secdown
\secp{\linux}

\begin{verbatim}
ponyatov@debian:~$ tar zx< Download/jdk-8u131-linux-x64.tar.gz
ponyatov@debian:~$ ls -la | grep jdk
drwxr-xr-x  8 ponyatov ponyatov   4096 мар 15 12:35 jdk1.8.0_131
ponyatov@debian:~$ env | grep jdk
PATH=/home/ponyatov/jdk1.8.0_131/bin:/usr/local/bin:/usr/bin:..
JAVA_HOME=/home/ponyatov/jdk1.8.0_131
ponyatov@debian:~$ grep .setenv ~/.profile ~/.xsessionrc
/home/ponyatov/.profile:. ~/.setenv
/home/ponyatov/.xsessionrc:. ~/.setenv

ponyatov@debian:~$ cat ~/.setenv
export JAVA_HOME="/home/ponyatov/jdk1.8.0_131"
export PATH="$JAVA_HOME/bin:$PATH"
\end{verbatim}

\secp{\win}

\begin{enumerate}[nosep]
  \item run installer: \menu{jdk-8u131-windows-i586.exe}
  \item select compiler path: \menu{JDK>C:/Java/jdk}
  \item select runtime path: \menu{JRE>C:/Java/jre}
  \item \menu{\winstart\ Start>Computer>\rms>System properties}
  \item \menu{Environment variables>User variables}
  \item \menu{New>JAVA\_HOME>C:/Java/jdk}
  \item \menu{New>PATH>\%JAVA\_HOME\%/bin;\%JAVA\_HOME\%/../jre/bin}
\end{enumerate}

\bigskip
Check install: \menu{\winr>cmd}
\begin{verbatim}
C:\Users\ponyatov> javac -version
java version 1.8.0_131
C:\Users\ponyatov> java -version
Java(TM) SE Runtime Environment (build 1.8.0_131-3464)
\end{verbatim}

\secup

\clearpage

\secrel{Seco}

\begin{enumerate}[nosep]
  \item \url{https://github.com/bolerio/seco/releases/latest}
  \item \menu{seco-dist-0.7.0.tgz} \linux\\\menu{seco-dist-0.7.0.zip} \win
\end{enumerate}

\secp{\linux}

\begin{verbatim}
ponyatov@debian:~$ tar zx < Download/seco-dist-0.7.0.tgz
ponyatov@debian:~$ ~/seco/run.sh 
\end{verbatim}

\secp{\win}

\begin{verbatim}
cd C:/Java
unzip C:/Users/ponyatov/Download/seco-dist-0.7.0.zip
C:/Java/seco/run.cmd
\end{verbatim}

\fig{/tmp/seco1.pdf}{width=\textwidth}

\noindent
If you fonts looks cut downed by an ax, you can fix \file{seco/run.sh}:
\begin{verbatim}
JAVA_EXEC='java -Dawt.useSystemAAFontSettings=lcd'
\end{verbatim}
\fig{/tmp/seco2.pdf}{height=0.8\textheight}

\secup

\secup

\end{document}
