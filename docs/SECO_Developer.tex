% Universal LaTeX headers for e-book publications
\documentclass[oneside,10pt]{book}
%% mobile phone optimized papaer size
\usepackage[paperwidth=118.8mm,paperheight=68.2mm,margin=2mm]{geometry}
%% hyperlinks pdf style
\usepackage[unicode,colorlinks=true]{hyperref}
%% fix first title page
\usepackage{atbegshi}\AtBeginDocument{\AtBeginShipoutNext{\AtBeginShipoutDiscard}}
%% font setup for screen reading
\renewcommand{\familydefault}{\sfdefault}\normalfont
%% fix heading styles for tiny paper
\usepackage{titlesec}
\titleformat{\chapter}{\Large\bfseries}{\thechapter.}{1em}{}
\titleformat{\section}{\large\bfseries}{\thesection.}{1em}{}
%% fix contents style
\usepackage{tocloft}
\renewcommand{\cfttoctitlefont}{\Large}
\renewcommand{\cftbeforetoctitleskip}{0mm}
\renewcommand{\cftaftertoctitleskip}{1em}
\renewcommand{\cftchapfont}{\bfseries}
\renewcommand{\cftbeforechapskip}{0.2em}
\renewcommand{\cftsecfont}{}
\renewcommand{\cftsubsecfont}{\it}
\renewcommand{\cftparskip}{0mm}
%% fix heading styles for tiny paper
\usepackage{titlesec}
\titleformat{\chapter}{\Large\bfseries}{\thechapter.}{1em}{}
\titlespacing{\chapter}{0em}{0em}{0.5em}
\titleformat{\section}{\large\bfseries}{\thesection.}{1em}{}
\titlespacing{\section}{0em}{1em}{0.5em}
\titleformat{\subsection}{\large\bfseries}{\thesubsection.}{1em}{}
\titlespacing{\subsection}{0em}{1em}{0.5em}
%% redefine colors, make it darker
\usepackage{xcolor}
\definecolor{red}{rgb}{0.7,0,0}		% R
\definecolor{green}{rgb}{0,0.6,0}	% G
\definecolor{blue}{rgb}{0,0,0.7}	% B

% images
\usepackage[pdftex]{graphicx}
\newcommand{\fig}[2]{\noindent\includegraphics[#2]{#1}}

% relative sectioning
\usepackage{ifthen}
\newcounter{secdepth}\setcounter{secdepth}{0}
\newcommand{\secup}{\addtocounter{secdepth}{1}}
\newcommand{\secdown}{\addtocounter{secdepth}{-1}}
\newcommand{\secrel}[1]{
\ifthenelse{\equal{\value{secdepth}}{0}}{\part{#1}}{}
\ifthenelse{\equal{\value{secdepth}}{-1}}{\chapter{#1}}{}
\ifthenelse{\equal{\value{secdepth}}{-2}}{\section{#1}}{}
\ifthenelse{\equal{\value{secdepth}}{-3}}{\subsection{#1}}{}
\ifthenelse{\equal{\value{secdepth}}{-4}}{\subsubsection{#1}}{}
}
\newcommand{\secp}[1]{\paragraph{#1}}
\newcommand{\secly}[1]{
\section*{#1}
\addcontentsline{toc}{section}{#1}
}

% [nosep] option in lists/enums
\usepackage{enumitem}
% frame box
\usepackage{framed}

% misc
\newcommand{\email}[1]{$<$\href{mailto:#1}{#1}$>$}
\renewcommand{\emph}[1]{\textcolor{red}{#1}}
\newcommand{\note}[1]{\,\footnote{\ #1}}
\newcommand{\term}[1]{\textcolor{green}{#1}}

% listing & computer
\usepackage[os=win]{menukeys}
\usepackage{amssymb} % windows key
\newcommand{\winstart}{$\boxplus$}
\newcommand{\winr}{\keys{\winstart+R}}
\newcommand{\lms}{$\lhd$}
% \newcommand{\dblms}{$\lhd\lhd$}
\newcommand{\rms}{$\rhd$}
% \newcommand{\checkbox}{$\boxtimes$}
% \newcommand{\uncheckbox}{$\square$}

%% languages
\newcommand{\seco}{$Seco$}
\newcommand{\hgdb}{$HyperGraph_{db}$}
\newcommand{\java}{$Java$}

%% OSes
\newcommand{\linux}{$Linux$}
\newcommand{\win}{$Windows$}

%% objects
\newcommand{\file}[1]{\textbf{#1}}
\newcommand{\prog}[1]{\textcolor{blue}{#1}}

%% std.tools
\newcommand{\git}{\prog{git}}
\newcommand{\make}{\prog{make}}

%%%%%%%%%%%%%%%%%%%%%%%%%%%%%%%%%%%%%%%%%%%%%%%%%%%%%%%%%%%%%%%%%%%%%%%%

\author{\copyright\ Dmitry Ponyatov \email{dponyatov@gmail.com}}
\title{\ \\\ \\Become a Seco Developer\\newbie guide}
\begin{document}
\maketitle
\tableofcontents\secdown
\clearpage

\secly{Intro to Seco IDE}

\seco\ is a collaborative scripting development environment for the \java\
platform. You can write code in many JVM scripting languages. The code editor in
Seco is based on the Mathematica notebook UI, but the full GUI is richer and
much more ambitious. In a notebook, you can mix rich text with code and output,
including interactive components created by your code. This makes Seco into a
live environment because you can evaluate expression and immediately see the
changes to your program.

\begin{description}
\item[GitHub]\ \\
\url{https://github.com/bolerio/seco}
\item[binary release builds]\ \\
\url{https://github.com/bolerio/seco/releases/latest}
\end{description}

You can use \seco\ from prebuild binary archive, but this manual will guide you
to become a developer able to expand \seco\ in ways you want.

\bigskip
This manual appeared when I found
\href{http://www.hypergraphdb.org/}{hypergraphdb} usable in my AI and knowledge
database research. \seco\ was noted as visual tool for experimenting with \hgdb,
and I was impressed by it's compact design. But \seco

\begin{description}[nosep]
\item[\href{https://github.com/bolerio/seco/issues/46}{issue 46}]
lacks of extension for \href{http://xsb.sourceforge.net/}{XSB Prolog} and
\href{http://coherentknowledge.com/comparison-of-ergo-suite-to-flora-2/}{Ergo/Flora}
system and
\item[\href{https://github.com/bolerio/seco/issues/45}{issue 45}]
has some problems and low usability with visual interface.
\end{description}

\bigskip
So I started some development without any notable experience \java\ development,
and add this manual going this way. You can track me on github, but note this is
fork project, and my modifications may be merged into core project:
\url{https://github.com/ponyatov/seco}

\secrel{Fresh taste}\secdown

For first time let's play with prebuild binary release: you will get feel of
Seco without long SDK installation.

\secrel{Workbench install} \secdown

\secrel{\java\ SDK}

\seco\ requires only \java\ RE, but we will install SDK for later use:

\begin{enumerate}[nosep]
\item \url{http://www.oracle.com/technetwork/java/javase/downloads/}
\item \menu{JDK>Download}
\item \menu{Java SE Development Kit 8u131>Accept License Agreement}
\item \menu{jdk-8u131-windows-i586.exe} 32 bit (win32)\\
\menu{jdk-8u131-linux-x64.tar.gz} 64 bit
\end{enumerate}

\secdown
\secp{\linux}

\begin{verbatim}
ponyatov@debian:~$ tar zx< Download/jdk-8u131-linux-x64.tar.gz
ponyatov@debian:~$ ls -la | grep jdk
drwxr-xr-x  8 ponyatov ponyatov   4096 мар 15 12:35 jdk1.8.0_131
ponyatov@debian:~$ env | grep jdk
PATH=/home/ponyatov/jdk1.8.0_131/bin:/usr/local/bin:/usr/bin:..
JAVA_HOME=/home/ponyatov/jdk1.8.0_131
ponyatov@debian:~$ grep .setenv ~/.profile ~/.xsessionrc
/home/ponyatov/.profile:. ~/.setenv
/home/ponyatov/.xsessionrc:. ~/.setenv

ponyatov@debian:~$ cat ~/.setenv
export JAVA_HOME="/home/ponyatov/jdk1.8.0_131"
export PATH="$JAVA_HOME/bin:$PATH"
\end{verbatim}

\secp{\win}

\begin{enumerate}[nosep]
  \item run installer: \menu{jdk-8u131-windows-i586.exe}
  \item select compiler path: \menu{JDK>C:/Java/jdk}
  \item select runtime path: \menu{JRE>C:/Java/jre}
  \item \menu{\winstart\ Start>Computer>\rms>System properties}
  \item \menu{Environment variables>User variables}
  \item \menu{New>JAVA\_HOME>C:/Java/jdk}
  \item \menu{New>PATH>\%JAVA\_HOME\%/bin;\%JAVA\_HOME\%/../jre/bin}
\end{enumerate}

\bigskip
Check install: \menu{\winr>cmd}
\begin{verbatim}
C:\Users\ponyatov> javac -version
java version 1.8.0_131
C:\Users\ponyatov> java -version
Java(TM) SE Runtime Environment (build 1.8.0_131-3464)
\end{verbatim}

\secup

\clearpage

\secrel{\seco\ prebuild binary distro}

\begin{enumerate}[nosep]
  \item \url{https://github.com/bolerio/seco/releases/latest}
  \item \menu{seco-dist-0.7.0.tgz} \linux\\\menu{seco-dist-0.7.0.zip} \win
\end{enumerate}

\secp{\linux}

\begin{verbatim}
ponyatov@debian:~$ tar zx < Download/seco-dist-0.7.0.tgz
ponyatov@debian:~$ ~/seco/run.sh 
\end{verbatim}

\secp{\win}

\begin{verbatim}
cd C:/Java
unzip C:/Users/ponyatov/Download/seco-dist-0.7.0.zip
C:/Java/seco/run.cmd
\end{verbatim}

\fig{../tmp/seco1.pdf}{width=\textwidth}

\noindent
If you fonts looks cut downed by an ax, you can fix \file{seco/run.sh}:
\begin{verbatim}
JAVA_EXEC='java -Dawt.useSystemAAFontSettings=lcd'
\end{verbatim}
\fig{../tmp/seco2.pdf}{height=0.8\textheight}

\secup

\secrel{Using \seco}\secdown

Some YouTube videos on using \seco:
  \href{https://www.youtube.com/watch?v=ktOzFNKrCpE}{Installation}
  
  You just have done this

\secrel{Notebook in Niche}  
  
  \href{https://www.youtube.com/watch?v=09Te_nSyHUA}{Notebooks}
  idea goes from interface of well-known computational and analytical
  math packages, preferrably Mathematica, and others\ --- Maxima, MathCAD,
  Axiom,\ldots
  
  Each \term{notebook}\ looks like linear document, composed from \term{cell}s.
  Cell can contain formatted text, code on some scripting language\note{code in
  Mathematica/Maxima syntax for example}, result of previous script cell
  execution, plot, image, interactive form,\ldots
  Cells can be grouped and folded/unfolded to hide information not used in this
  moment.
  
  The cell types are essentially three: \emph{input}, \emph{output} and
  \emph{documentation} cells.
  
  \seco\ uses some sort of graph database for storing data, so it uses
  \term{niche}s to group and store notebooks and other working data.
  You can think about \term{niche}\ as some sort of \emph{knowledge database}.
  Starting \seco\ for first time you can get dialog asking of new niche
  creation. In this case you should select storage directory including niche
  name as last name element. If you have no dialog on \seco\ start, look on
\begin{verbatim}
ponyatov@debian:~$ ls -la ~ |grep .seco
drwxr-xr-x  2 ponyatov ponyatov   .secoDefaultNiche
drwxr-xr-x  2 ponyatov ponyatov   .secoRepository

.secoDefaultNiche/:
00000000.jdb  hgdbversion  je.info.0  je.lck

.secoRepository/:
00000000.jdb  hgdbversion  je.info.0  je.lck
\end{verbatim}

\noindent  
Create new notebook for experiments: \menu{Notebook>New} or \keys{Ctrl+N}

\noindent  
You'll get new empty notebook tab \menu{CG},\\
rename it \menu{CG>\rms>Rename>Tutorial}
\bigskip

Now click in empty notebook, you will see horisontal line, points to current
entry cell. Start typing \menu{hello}, and you will see new cell.

\bigskip
\fig{../tmp/hello.pdf}{width=0.8\textwidth}\\
By default \seco\ creates script cells using \java-like scripting language
\href{http://www.beanshell.org/manual/quickstart.html}{beanshell}

\clearpage
Cell can be evaluated in place: press \keys{Shift+Enter} when your cursor 
in cell.
\begin{verbatim}
hello

null
\end{verbatim}

Hmm, something strange, fix it:

\begin{verbatim}
"hello"

hello
\end{verbatim}

So, \verb$"text"$ in double quots is \term{string} literal. BeanShell
interpreter evaluates it as string and puts into output cell as is.

\clearpage\noindent
You ever can use GUI elements in output:
\fig{../tmp/hellobtn.pdf}{height=0.3\textheight}

\bigskip
For documenting your code you must have ability to make just formatted text,
\seco\ provides this by changing input cell type to HTML. Move your cursor to
the beginning of notebook by \keys{Ctrl+Home} and input:
\begin{verbatim}
This demo code will be used in SECO_Developer tutorial
\end{verbatim}

Then press \keys{Ctrl+Space} and select \menu{html} input syntax. Cell becomes
html cell, it is unevaluable by \keys{Shift+Enter}, but can be viewed in
\emph{html} and \emph{source} form. Use \menu{\rms>Source}, and add header tags:
\begin{verbatim}
<H1>This demo code will be used in SECO_Developer tutorial</H1>
\end{verbatim}
So you can use some basic hmtl tags to format your text, using WISIWIG editor
elements in \emph{html} mode, or by direct tag input in \emph{source} mode.
When you want to switch back into html mode, use \menu{\rms>Cell>Html View}.
Text too large, select it in html view mode, and use drop-down selector in
menu bar to change text style into smaller \verb$h2$ title.
 
\bigskip
\fig{../tmp/htitle.pdf}{width=0.95\textwidth}
  
\secrel{User Interface}
 
  \href{https://www.youtube.com/watch?v=k9fbOmc2tOk}{User Interface}
  
\secrel{External Libraries Integration}

\href{https://www.youtube.com/watch?v=2AM85plUtDU}{Runtime}

\secup

\secrel{\hgdb\ intro}

\secup

\secrel{Hacking \seco}\secdown

If you want more complex system, It's time to deep into source.

\secrel{Workbench install}\secdown

\secrel{\git\ client}

\secp{\linux}

\begin{verbatim}
ponyatov@debian:~$ sudo apt install git
[sudo] password for ponyatov: xxxxxxxx
Reading package lists... Done
Building dependency tree
...
\end{verbatim}

\secp{\win}

\secrel{\seco\ source}

\secup


\secup

\end{document}
