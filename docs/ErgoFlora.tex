\secrel{\ergo/\flora}\secdown

\ergo\ and it's free variant \flora\ --- the most powerful implementation of
\prolog\ programming language, expanded with lot of modern logic programming
features: memoization, tabling, frames, datalog, natural language processing,
machine learning, semantic web,\ldots \ergo/\flora\ uses \xsb\ as it's core.

\begin{itemize}[nosep]
  \item \ergo\ is commercial full-featured system developed by\\
  \href{http://coherentknowledge.com/product-overview-ergo-suite-platform/}{Coherent
  Knowledge}
  \item \flora\ is free lightweight variant at\\
  \url{http://flora.sourceforge.net/}
  \item \xsb\ itself available for free\\
  \url{http://xsb.sourceforge.net/}
  \item \href{http://interprolog.com/java-bridge/}{\prog{InterProlog}} is
  \java-bridge for \xsb
\end{itemize}

\secrel{\flora\ Install}

\begin{enumerate}[nosep]
  \item 
\menu{\url{http://flora.sourceforge.net/}>Download>\href{https://sourceforge.net/projects/flora/files/FLORA-2/}{latest
release}}
\end{enumerate}

\secp{\linux}\ \\

\begin{itemize}[nosep]
  \item 
\menu{Download x.x (Monstera deliciosa) \linux/Mac/Unixes>\file{flora2.run}}
\begin{verbatim}
ponyatov@debian:~$ chmod +x Download/flora2.run
ponyatov@debian:~$ Download/flora2.run --target ~/FLORA
\end{verbatim}
\end{itemize}

\secp{\win}\ \\

\begin{itemize}[nosep]
  \item 
\menu{Download x.x (Monstera deliciosa) \win>\file{Flora-2.exe} (installer)}
  \item 
\menu{Install to \file{D:/FLORA}}
\end{itemize}

\secrel{\xsb\ integration}

Binary \flora\ distribution from upper section goes preloaded with \xsb\ and
\java Bridge so you don't need to install anything specially for integration
with \seco. But you need to add extra .jars into your \seco\ working
configuration:

\bigskip
\begin{enumerate}[nosep]
  \item \menu{\seco>Notebook>New>CG>\rms>Rename>FLORA}
  \item \menu{\seco>Runtime>Configure Current>Add ClassPath Entry}
  \item \menu{\file{file:/home/ponyatov/FLORA/flora2/java/interprolog.jar}}
\end{enumerate}

\lstx{\menu{FLORA>Init cell}}{src/interprolog.seco}{Java}

\secrel{\prolog\ intro}\secdown

The best tutorial on using \prolog\ you can found is
\href{https://www.cpp.edu/~jrfisher/www/prolog_tutorial/contents.html}{J.R.Fisher's
tutorial}.

\bigskip
\prolog\ can be very complex to understand especially for people already have
some experience in other imperative or functional language. The key let you
understand prolog\ --- think about clauses like hypergraph elements:
\begin{itemize}
  \item 
every term defines named relation between it's arguments.
\end{itemize}

\clearpage
For first steps you can run \flora\ directly from command line, and load sample
files using
\lst{direct \flora\ run from command line \linux}{src/linflora.rc}
\lst{direct \flora\ run from command line \win}{src/winflora.rc}

\secrel{Load a program from a local file}

\lst{(re)load this from local \file{.flr}\ file
using command \file{[Hello].}}{src/hello.flr}

\begin{verbatim}
flora2 ?- [Hello].
Hello World

Times (in seconds): elapsed = 0.0000; pure CPU = 0.0000

Yes
\end{verbatim}

On \win\ you can use \keys{\arrowkeyup}\ to reload last \verb$[Hello].$

\secrel{.init file}

\secrel{Comments}

\flora\ uses \cpp-like comments:
\begin{verbatim}
// line comment
/* block comment */
\end{verbatim}

\secup


\secup
