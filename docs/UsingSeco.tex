\secrel{Using \seco}\secdown

Some YouTube videos on using \seco:
  \href{https://www.youtube.com/watch?v=ktOzFNKrCpE}{Installation}
  
  You just have done this

\secrel{Notebook in Niche}
  
  \href{https://www.youtube.com/watch?v=09Te_nSyHUA}{Notebooks}
  idea goes from interface of well-known computational and analytical
  math packages, preferrably Mathematica, and others\ --- Maxima, MathCAD,
  Axiom,\ldots
  
  Each \term{notebook}\ looks like linear document, composed from \term{cell}s.
  Cell can contain formatted text, code on some scripting language\note{code in
  Mathematica/Maxima syntax for example}, result of previous script cell
  execution, plot, image, interactive form,\ldots
  Cells can be grouped and folded/unfolded to hide information not used in this
  moment.
  
  The cell types are essentially three: \emph{input}, \emph{output} and
  \emph{documentation} cells.
  
  \seco\ uses some sort of graph database for storing data, so it uses
  \term{niche}s to group and store notebooks and other working data.
  You can think about \term{niche}\ as some sort of \emph{knowledge database}.
  Starting \seco\ for first time you can get dialog asking of new niche
  creation. In this case you should select storage directory including niche
  name as last name element. If you have no dialog on \seco\ start, look on
\begin{verbatim}
ponyatov@debian:~$ ls -la ~ |grep .seco
drwxr-xr-x  2 ponyatov ponyatov   .secoDefaultNiche
drwxr-xr-x  2 ponyatov ponyatov   .secoRepository

.secoDefaultNiche/:
00000000.jdb  hgdbversion  je.info.0  je.lck

.secoRepository/:
00000000.jdb  hgdbversion  je.info.0  je.lck
\end{verbatim}

\noindent  
Create new notebook for experiments: \menu{Notebook>New} or \keys{Ctrl+N}

\noindent  
You'll get new empty notebook tab \menu{CG},\\
rename it \menu{CG>\rms>Rename>Tutorial}
\bigskip

Now click in empty notebook, you will see horisontal line, points to current
entry cell. Start typing \menu{hello}, and you will see new cell.

\bigskip
\fig{fig/hello.png}{width=0.8\textwidth}\\
By default \seco\ creates script cells using \java-like scripting language
\href{http://www.beanshell.org/manual/quickstart.html}{beanshell}

\clearpage
Cell can be evaluated in place: press \keys{Shift+Enter} when your cursor 
in cell.
\begin{verbatim}
hello

null
\end{verbatim}

Hmm, something strange, fix it:

\begin{verbatim}
"hello"

hello
\end{verbatim}

So, \verb$"text"$ in double quots is \term{string} literal. BeanShell
interpreter evaluates it as string and puts into output cell as is.

\clearpage\noindent
You ever can use GUI elements in output:
\fig{fig/hellobtn.png}{height=0.3\textheight}

\bigskip
For documenting your code you must have ability to make just formatted text,
\seco\ provides this by changing input cell type to HTML. Move your cursor to
the beginning of notebook by \keys{Ctrl+Home} and input:
\begin{verbatim}
This demo code will be used in SECO_Developer tutorial
\end{verbatim}

Then press \keys{Ctrl+Space} and select \menu{html} input syntax. Cell becomes
html cell, it is unevaluable by \keys{Shift+Enter}, but can be viewed in
\emph{html} and \emph{source} form. Use \menu{\rms>Source}, and add header tags:
\begin{verbatim}
<H1>This demo code will be used in SECO_Developer tutorial</H1>
\end{verbatim}
So you can use some basic hmtl tags to format your text, using WISIWIG editor
elements in \emph{html} mode, or by direct tag input in \emph{source} mode.
When you want to switch back into html mode, use \menu{\rms>Cell>Html View}.
Text too large, select it in html view mode, and use drop-down selector in
menu bar to change text style into smaller \verb$h2$ title.
 
\bigskip
\fig{fig/htitle.png}{width=0.95\textwidth}
  
\secrel{User Interface}
 
By default \seco\ runs in tabbed \term{simpleUI} mode. If you want to run it in
mode presented on \href{https://www.youtube.com/watch?v=k9fbOmc2tOk}{this
video}, you should run command line:
\begin{verbatim}
ponyatov@debian:~$ ~/seco/run.sh --simpleUI false &
\end{verbatim}
But this mode was broken in current binary release build (0.7.0).\\
And this is first point to apply some fixes in source code.

\bigskip
Current \seco\ has very limited support of interface for graph manipulation and
data visualization. Lot of work must be applied to expanding interface with
science data vizualization (ParaView bindings), computer science algorithms and
data structures (first of all multilanguage compiler\&translator design and
dynamic memory management), and CAD/CAM visual elements for engineering
applications.

\secup


