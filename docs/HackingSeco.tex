\secrel{Hacking \seco}\secdown

If you want more complex system, It's time to deep into source.

\secrel{Workbench install}\secdown

\secrel{\git\ client for \github}

\secp{\linux}

\begin{verbatim}
ponyatov@debian:~$ sudo apt install git
[sudo] password for ponyatov: xxxxxxxx
Reading package lists... Done
Building dependency tree
...
\end{verbatim}

\secp{\win}

\url{https://git-scm.com/download/win}

\secp{Register developer account at \github}\ \\

\noindent
If you still have no developer account at \href{https://github.com/}{\github},
you should create it.

\bigskip
\begin{itemize}[nosep]
\item \menu{\href{https://github.com/}{\github}>right-up corner>Sign up}
\item \menu{Username>ponyatov}
\item \menu{Email Address>dponyatov@gmail.com}
\item \menu{Password>*********}
\item \keys{Create an account}
\item \menu{Choose your plan>\checkbox\ Unlimited public repositories for
free>Continue}
\item Verify mailbox by clicking on link in email
\item go to your github account \menu{\url{https://github.com/ponyatov/}}
 
\end{itemize}

\secp{Generating SSH keys}\ \\

\noindent
For \emph{developer} use of \github\ you \emph{must} create and register crypto
keys, as described at
\href{https://help.github.com/articles/connecting-to-github-with-ssh/}{SSH
help}:

\bigskip
\begin{itemize}[nosep]
  \item click in drop-down menu in right-up corner of your github homepage
  \item select \menu{Settings>Personal settings>SSH and GPG keys}
  \item \menu{SSH keys>New SSH key}
  \item open man in new browser tab:
  \href{https://help.github.com/articles/generating-an-ssh-key/}{generating SSH
  keys}
  \item
  \href{https://help.github.com/articles/generating-a-new-ssh-key-and-adding-it-to-the-ssh-agent/}{Generating
  a new SSH key and adding it to the ssh-agent}
\end{itemize}
\begin{verbatim}
ssh-keygen -t rsa -b 4096 -C "bvsaancr@sharklasers.com"
Enter a file ... save the key (/home/you/.ssh/id_rsa): [Enter]
Enter passphrase (empty for no passphrase): [Type a passphrase]
Enter same passphrase again: [Type passphrase again]
\end{verbatim}
\begin{itemize}[nosep]
  \item open file \verb$~/.ssh/id_rsa.pub$ in any text editor and copy it's
  content into clipboard
  \item go back to internet browser with \menu{New SSH key} form
  \item \menu{Title>Debian Linux developer key}
  \item \menu{Key>paste \file{\~{}/.ssh/id\_rsa.pub} contents here}
  \item \keys{Add SSH key}
\item You'll see your new key in list
\end{itemize}
\begin{verbatim}
Debian Linux developer key
Fingerprint: e8:ba:1d:f4:67:ce:02:b9:c8:ee:23:09:e8:1b:b2:ff
\end{verbatim}
\begin{itemize}[nosep]
  \item check your new key by
  \href{https://help.github.com/articles/testing-your-ssh-connection/}{Testing
  your SSH connection} recipe:
\begin{verbatim}
ponyatov@debian:~$ ssh -T git@github.com
Hi ponyatov! You've successfully authenticated,
but GitHub does not provide shell access.
\end{verbatim}
\end{itemize}

\secp{Register SSH keys in \prog{ssh-agent} (\linux\ only)}\ \\

\noindent
Every time you relogin into your \linux\ system, you \emph{must} open terminal
and \emph{register SSH key} in \prog{ssh-agent}:
\begin{verbatim}
ponyatov@debian:~$ ssh-add
Enter passphrase for /home/ponyatov/.ssh/id_rsa: 
Identity added: /home/ponyatov/.ssh/id_rsa (...)
ponyatov@debian:~$ cd seco
ponyatov@debian:~/seco$ git gui &
\end{verbatim}
I still have no idea how to make it automatically from dotfiles, but this step
\emph{required} for \prog{git gui} work when you push local changes into
\github\ repository. If you forgot it, your \prog{git push} in GUI will hangs.
To clean-up this error use 
\begin{verbatim}
ponyatov@debian:~/seco$ killall -9 wish
ponyatov@debian:~/seco$ ssh-add
\end{verbatim}

\secrel{Fork \seco\ source}

\noindent
To do your own development using somebody's project on \github, you \emph{must}
\term{fork} it into \emph{your} \github\ account.

\begin{itemize}[nosep]
  \item go to master \seco\ repository:
  \url{https://github.com/bolerio/seco/}
  \item or my\note{preferrred, I need some practice to write
  section on right pull requesting}\ developer fork
  \url{https://github.com/ponyatov/seco/}
  \item press \keys{Fork}\ button in right up corner
  \item \menu{Where should we fork this repository?>select your account}
  \item Forking ponyatov/seco: it should be some seconds later\ldots 
\end{itemize}
Now you see your new \term{repository}:
\begin{verbatim}
kyjbfjqx/seco
forked from ponyatov/seco
Branch: master
This branch is even with ponyatov:master.
\end{verbatim}
\begin{framed}
Your \term{master} branch must be equal with \emph{upstream} project tree
\end{framed}
This requirement is determined by the need to track all changes made by main
developer or by project maintainers. Tiny projects ordinarily use only
\term{master} branch, more used and large projects use \term{development} branch
and leave master branch for \term{stable releases}.
\begin{framed}
If main team \term{approves} some changes in project, you can pull/merge this
changes into \emph{your} master branch, and then check that all your code in
other branches is compatible with core project evolution.
\end{framed}
Do not do any changes in your master branch\ --- it is reserved exclusively for
origin$\rightarrow$fork changes.
\clearpage

If you forked \emph{my} repo, you also have some branches like \prog{manual},
\prog{fontscale},\ldots\\It is good idea do not override them too to track
fork$\rightarrow$subfork changes. So if you do not plan to do some megaproject,
right way to track your changes is creating unical branches for tiny specific
problems, and do all work in sync with whole developer's team (for example open
bugreports and fix \href{https://github.com/bolerio/seco/issues}{issues
registered in core project}).

\bigskip
Most basics of collaborative work on \github\ covered by this manuals:
\begin{itemize}
  \item \href{https://guides.github.com/activities/forking/}{Forking Projects}
\end{itemize}

\clearpage
\secrel{Clone your \github\ repo locally}

\begin{enumerate}[nosep]
  \item move binary seco to separate folder
\begin{verbatim}
ponyatov@debian:~$ mv seco seco.bin
\end{verbatim} 
\item clone \github\ repo locally into your home dir
\begin{itemize}[nosep]
  \item go to your \url{https://github.com/kyjbfjqx/seco}
  \item click \menu{Clone or download} green button
\item \menu{Clone with>Use SSH>Copy to clipboard}
\end{itemize}
\begin{verbatim}
kyjbfjqx@debian:~$ git clone -o gh --depth=1 \
    git@github.com:kyjbfjqx/seco.git seco
seco Cloning into 'seco'...
remote: Counting objects: 2665, done.
remote: Compressing objects: 100% (1574/1574), done.
Receiving objects:  91% (2432/2665), 14.05 MiB | 926.00 KiB/s   
\end{verbatim}
\item go to cloned repository and move to learning branch  
\begin{verbatim}
kyjbfjqx@debian:~$ cd seco/
kyjbfjqx@debian:~/seco$ git remote add ponyatov \
    git@github.com:ponyatov/seco.git
kyjbfjqx@debian:~/seco$ git checkout -b learning
kyjbfjqx@debian:~/seco$ git branch
* learning
  master
kyjbfjqx@debian:~/seco$ git pull ponyatov learning
kyjbfjqx@debian:~/seco$ git push gh learning
\end{verbatim}
\clearpage
\item create branch using
\href{https://github.com/bolerio/seco/issues/52}{issue} registered on
\href{https://github.com/bolerio/seco/issues}{upstream tracking system}
\begin{verbatim}
kyjbfjqx@debian:~/learning$ git checkout -b issue52 
Switched to a new branch 'issue52'
kyjbfjqx@debian:~/learning$ git branch
* issue52
  learning
  master
kyjbfjqx@debian:~/learning$ git push -u gh issue52
Total 0 (delta 0), reused 0 (delta 0)
To git@github.com:kyjbfjqx/seco.git
 * [new branch]      issue52 -> issue52
Branch issue52 set up to track remote issue52 from gh.
\end{verbatim}
Now you can update \github\ project page in browser and select to your new
branch \menu{Branch>issue52}
\end{enumerate}
 
\clearpage
\secrel{IDE \eclipse}

\noindent
\eclipse\ is heavy resource IDE intensively uses JVM, so you need some more or
less modern computer to run it.

\begin{enumerate}[nosep]
  \item \url{https://www.eclipse.org/downloads/eclipse-packages/}
  \item Later we will widely use autogenerated code in \cpp\ so download\\
  \menu{Eclipse IDE for C/C++ Developers>\bit{32/64}}
  \item install \eclipse\ distro to any directory: just unpack .zip to\\
  \menu{/home/ponyatov/eclipse} or \menu{C:/Java/eclipse}
\item on first IDE start select path for your working project storage:\\
\menu{/home/ponyatov/} or \menu{D:/w}
  \item create empty generic project 
\menu{\keys{Ctrl+N}>General>Project>hello}
\item create new empty .java file by
\menu{\keys{Ctrl+N}>General>File>java.java}
\item confirm search in \eclipse\ Marketplace on unknown file extension
\item install \menu{Marketplace>Eclipse Java Development Tools}
\item preconfigure all source code editors\\
\menu{Window>Preferences>General>Editors}\\
\menu{Text Editors>\checkbox\ show print margin>80}\\
this will help you to  make text files able to view on 80 chars VGA console
\end{enumerate}

\secrel{\gvim\ editor (IDE fallback)}

\noindent
If you computer not so powerful, or you need to do fast fix into any text file
without waiting \eclipse\ startup, you can install \gvim\ text editor, can
be treated and tuned as light IDE.

\secp{\linux}

\begin{verbatim}
sudo apt install vim-gui-common
\end{verbatim}

\secp{\win}\ \\

Install from \url{http://www.vim.org/download.php#pc}

\bigskip
After \gvim\ installation you will be able to edit any file by \menu{\rms>Edit
in vim} in file explorer, or use command line \verb$gvim any.file$.

\secp{vimrc}\ \\

Add this config file to your \verb|$HOME$|, it will change \gvim\ behaviour.
\lstx{\file{C:/Users/ponyatov/\_vimrc} or
\file{\~{}/.vimrc}}{src/vim.rc}{sh}

\secup

\clearpage
\secrel{Hacking \hgdb}\secdown

We will not participate in the development of \hgdb\note{this project too large
to spend lot of time on it}, but we will use it's sources to rebuild some .jars
in \seco\ from source, and we need some extra tools like \hgv. 

\secrel{Clone sources}

\lstx{\prog{git clone} from \github\ (read-only over
\file{https://})}{src/hgdbclone.rc}{sh}

\secrel{Build using \file{Makefile}}

And in this point I see then we still can do some fixes: build \hgdb\ using
\make: \href{https://github.com/hypergraphdb/hypergraphdb/issues/128}{issue128}

\lstx{fork \hgdb\ and replace local source tree}{src/hgdbsrc.rc}{sh}
 

\secup

\secup

